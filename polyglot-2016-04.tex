\documentclass[mathserif,xcolor=svgnames]{beamer}
\usepackage{colortbl}
\usepackage{listings}
\usepackage{color}
\usepackage[T1]{fontenc}
\usepackage{textcomp}

\definecolor{dkgreen}{rgb}{0,0.6,0}
\definecolor{gray}{rgb}{0.5,0.5,0.5}
\definecolor{mauve}{rgb}{0.58,0,0.82}

\lstset{frame=tb,
  aboveskip=3mm,
  belowskip=3mm,
  showstringspaces=false,
  columns=fullflexible,
  %basicstyle=\footnotesize,
  basicstyle={\small\ttfamily},
  numbers=none,
  numberstyle=\tiny\color{gray},
  keywordstyle=\color{blue},
  commentstyle=\color{dkgreen},
  stringstyle=\color{mauve},
  breaklines=true,
  breakatwhitespace=true,
  language=Caml,
  morekeywords={type,val,let,in,rec,function},
  tabsize=4,
  upquote=true
}

%\usepackage{beamerthemeAmsterdam} % Non-standard. Find on Google.
\usepackage{beamerleanprogress} % Non-standard. Find on Google.
  \renewcommand{\emph}[1]{{\em #1}}
%\usecolortheme{seahorse} % lighter colours
\setbeamertemplate{navigation symbols}{} % remove silly navigation symbols
%\setbeameroption{show notes}

% Don't include backup slides in page numbers; see
% stackoverflow.com/questions/732902/ignoring-page-numbers-in-backup-slides
\newcommand{\backupbegin}{
   \newcounter{backupslides}
   \setcounter{backupslides}{\value{framenumber}}
}
\newcommand{\backupend}{
   \addtocounter{backupslides}{-\value{framenumber}}
   \addtocounter{framenumber}{\value{backupslides}}
}

\usepackage{amsmath}
  % Slanted captical Greek letters
  \let\upPi\Pi
  \let\upSigma\Sigma
  \let\Gamma\varGamma
  \let\Delta\varDelta
  \let\Theta\varTheta
  \let\Lambda\varLambda
  \let\Xi\varXi
  \let\Pi\varPi
  \let\Sigma\varSigma
  \let\Upsilon\varUpsilon
  \let\Phi\varPhi
  \let\Psi\varPsi
  \let\Omega\varOmega

\usepackage{amssymb}
\usepackage{stmaryrd}
\usepackage{array,multirow}

% Specific to Beamer slides
\renewcommand{\_}{\mathunderscore}


%TODO:
\begin{document}

\title[OCaml]{\
  Ocaml
}
\author[Anders Fugmann]{\
  \normalfont \underline{Anders Fugmann}
}
\institute{Polyglot Meetup \#8
}
\date{}

\maketitle

\section{Introduction}

\begin{frame}
  \frametitle{Overview}

  \begin{enumerate}
  \item What is Ocaml
  \item A small example
  \item Another short example
  \item Extended example
  \item Full scale
  \end{enumerate}
\end{frame}

\begin{frame}[fragile]
   \frametitle{What is OCaml}
   \begin{enumerate}
   \item Functional Programming Language
   \item Interpreted and nativly compiled (1/3 speed of C)
   \item Supports imperative constructs
   \end{enumerate}
\end{frame}


% Brief overview
% Fac
% Let bindings, curreied functions, high order functions
% Imperative constructs
% Concurrency (No paralism) - Like stackless python and javascript
% An example of an expression parser
% Go though the steps
% Show the output
% Concurrency (No paralism) - Like stackless python and javascript
% Create a web-server
% Explain what happends
% Talk about opam
% Testing - General thoughs on why functional programming is easier to
%   test
% Create a simple test suite


\begin{frame}[fragile]
  \frametitle{something}
\end{frame}

\begin{frame}[fragile]
  \frametitle{A small example}
  \begin{lstlisting}
  let rec fac = function
    | 0 -> 1
    | n -> n * fac (n - 1)
  \end{lstlisting}
\end{frame}

\begin{frame}[fragile]
  \frametitle{Language Constructs}
  \begin{itemize}
    \item strictly evaluated
    \item let bindings are aliases
    \item function arguments are seperated by space
    \item functions arguments are curried
    \item functions are first class
    \item tail call optimization
    \item no null pointers!
  \end{itemize}
\end{frame}

\begin{frame}[fragile]
  \frametitle{Lists}
  Lists are immutable.
  List are destructed with the '::' operator
  \begin{lstlisting}
    (* Destutter *)
    let rec destutter = function
      | [] -> []
      | [ x ] -> [ x ]
      | x :: y :: xs when x = y -> destutter (y :: xs)
      | x :: y :: xs -> x :: destutter (y :: xs)

    destutter [ 1; 3; 3; 5 ]

    (* Produces *)
    [ 1; 3; 5 ]
  \end{lstlisting}

\end{frame}


\begin{frame}[fragile]
  \frametitle{An Expression Evaluator}
  \lstinputlisting{polyparser.ml}
\end{frame}

\begin{frame}[fragile]
  \frametitle{An Extendeed Expression Evaluator}
  \lstinputlisting{polyparser_test.ml}
\end{frame}


\begin{frame}[fragile]
  \frametitle{AVL trees}
  \begin{lstlisting}
    (* Piano numbers: succ (succ (succ z)): 3 *)
    type z
    type succ
    type 'a s = succ * 'a

    type _ node =
      | LNode: ('a s node * 'a node   * int) -> 'a s s node
      | RNode: ('a node   * 'a s node * int) -> 'a s s node
      | BNode: ('a node   * 'a node   * int) -> 'a s node
      | Empty: z node
  \end{lstlisting}

\end{frame}

\section{}
\begin{frame}
  \center \LARGE Questions
\end{frame}

\end{document}


%% Need to create a webserver, using the API to actually produce
%% something

%% Mention that ocaml cannot branch on type runtime. Therefore we
%% need to tag our variables - like units.

%% Show opam - That there exists a community
%% Second section is about not representing the illegal state.

%% Redo first section
%% - Use current example to show all places where the illegal state is
%% representable, and fix it.
