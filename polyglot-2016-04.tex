\documentclass[xcolor=svgnames]{beamer}
\usepackage{colortbl}
\usepackage{listings}
\usepackage{color}
\usepackage[T1]{fontenc}
\usepackage{textcomp}

\definecolor{dkgreen}{rgb}{0,0.6,0}
\definecolor{gray}{rgb}{0.5,0.5,0.5}
\definecolor{mauve}{rgb}{0.58,0,0.82}

\lstset{frame=tb,
  aboveskip=3mm,
  belowskip=3mm,
  showstringspaces=false,
  columns=fullflexible,
  %basicstyle=\footnotesize,
  basicstyle={\small\ttfamily},
  numbers=left,
  numberstyle=\tiny\color{gray},
  keywordstyle=\color{blue},
  commentstyle=\color{dkgreen},
  stringstyle=\color{mauve},
  breaklines=true,
  breakatwhitespace=true,
  language=Caml,
  morekeywords={type,val,let,in,rec,function},
  tabsize=4,
  upquote=true
}

%\usepackage{beamerthemeAmsterdam} % Non-standard. Find on Google.
\usepackage{beamerleanprogress} % Non-standard. Find on Google.
  \renewcommand{\emph}[1]{{\em #1}}
%\usecolortheme{seahorse} % lighter colours
\setbeamertemplate{navigation symbols}{} % remove silly navigation symbols
%\setbeameroption{show notes}

% Don't include backup slides in page numbers; see
% stackoverflow.com/questions/732902/ignoring-page-numbers-in-backup-slides
\newcommand{\backupbegin}{
   \newcounter{backupslides}
   \setcounter{backupslides}{\value{framenumber}}
}
\newcommand{\backupend}{
   \addtocounter{backupslides}{-\value{framenumber}}
   \addtocounter{framenumber}{\value{backupslides}}
}

\usepackage{amsmath}
  % Slanted captical Greek letters
  \let\upPi\Pi
  \let\upSigma\Sigma
  \let\Gamma\varGamma
  \let\Delta\varDelta
  \let\Theta\varTheta
  \let\Lambda\varLambda
  \let\Xi\varXi
  \let\Pi\varPi
  \let\Sigma\varSigma
  \let\Upsilon\varUpsilon
  \let\Phi\varPhi
  \let\Psi\varPsi
  \let\Omega\varOmega

\usepackage{amssymb}
\usepackage{stmaryrd}
\usepackage{array,multirow}

% Specific to Beamer slides
\renewcommand{\_}{\mathunderscore}


%TODO:
\begin{document}

\title[Introduction to OCaml]{\
  OCaml
}
\author[Anders Fugmann]{\
  \normalfont \underline{Anders Fugmann}
}
\institute{Polyglot Meetup \#8
}
\date{}

\maketitle

\section{Introduction}

\begin{frame}[fragile]
  \frametitle{What is OCaml}
  \pause
   \begin{itemize}
   \item Functional Programming Language - But not pure!
  \pause
   \item Terse
  \pause
   \item Garbage collected
  \pause
   \item Fast
   \end{itemize}
\end{frame}

\begin{frame}[fragile]
  \frametitle{What is OCaml}
  \begin{center}
    \Huge It's fun
  \end{center}
\end{frame}

\begin{frame}[fragile]
  \frametitle{A small example}
  A Textbook example; factorial
  \begin{lstlisting}
  # let rec fac = function
    | 0 -> 1
    | n -> n * fac (n - 1);;
  val fac : ... = <fun>

  # fac 5;;
  - : ... = 120

  \end{lstlisting}
\end{frame}

\section{Language}
\begin{frame}
  OCaml Language
  \begin{itemize}
  \item Pattern matching
  \item Functions are first class.
  \item Function currying
  \item Anonymous functions and closures
  \item Algebraic Datatypes, Tuples and records
  \item Modules and Functors
  \item ...
  \end{itemize}
\end{frame}

\section{A Small Calculator}
\begin{frame}[fragile]
  \frametitle{A small calculator}
  A small calculator to evaluate expressions like $3 + 4$.
  \pause
  \lstinputlisting[firstline=23]{polyparser.ml}
\end{frame}

\begin{frame}[fragile]
  \frametitle{Running the code}
  \begin{lstlisting}
    $ ocamlfind ocamlc -package mparser -linkpkg \
      -o polyparser polyparser.ml
    $ ./polyparser 3+4
    7
    $ ./polyparser 3+4*7+1
    32
  \end{lstlisting}
\end{frame}

\begin{frame}[fragile]
  \frametitle{A small webserver}
  \lstinputlisting[firstline=32]{polyparser_http.ml}
  \ttfamily{http://localhost:8000/3+4}
\end{frame}

\begin{frame}[fragile]
  \frametitle{Compiling to javascript}
  OCaml can be compiled to javascript and has API's for DOM
  manipulation, js events, XHR, etc.

  \begin{lstlisting}
    $ js_of_ocaml polyparser -o polyparser.js
    $ nodejs polyparser.js 3+4
    7
  \end{lstlisting}
  \pause
  But the javascript code is not really readable.
\end{frame}

%%%%%%%%%%%%%%%%%%%%%%%
%Types
%%%%%%%%%%%%%%%%%%%%%%
\section{Testing}
\begin{frame}[fragile]
  \begin{center}
    \Huge Testing
  \end{center}
\end{frame}

\begin{frame}[fragile]
  \frametitle{Type Inference}
  Type inference is used to validate our assumptions in the code. This
  will find all the stupid mistakes and let us concentrate on the hard
  stuff.\newline\\
  \pause
  Type inference removes the need for exhaustive unit tests.\newline
  \pause
  \begin{itemize}
    \item OCaml has a ``rich'' type system
    \item Validates that the program is ``sound''
    \item Not explicitly typed!
  \end{itemize}
\end{frame}


\begin{frame}[fragile]
  \frametitle{Type Inference}
  Lets re-visit the previous examples:
  \begin{lstlisting}
  # let rec fac = function
    | 0 -> 1
    | n -> n * fac (n - 1);;
  val fac : int -> int = <fun>

  # fac 5;;
  - : int = 120

  # let f x y = x + y;;
  val f : int -> int -> int;;
  # 3 * "Ho";;
  Error: This expression has type string but an expression was expected of type int
  \end{lstlisting}

\end{frame}

\begin{frame}[fragile]
  \frametitle{A Simple Calculator}
  In the calculator example, we add a type to help the compiler
  understand our algebra
  \begin{lstlisting}
    type op = Add | Mul
    type expr =
      | Int of int
      | Binop of op * expr * expr

    let rec eval = function
      | Int n -> n
      | Binop (Add, a, b) -> eval a + eval b
   (* | Binop (Mul, a, b) -> eval a * eval b *)

  \end{lstlisting}
  \pause
  \begin{lstlisting}[language=make,frame=none,numbers=none]
    $ ocamlc adtparser.ml
    File "adtparser.ml", line 6, characters 15-79:
    Warning 8: this pattern-matching is not exhaustive.
    Here is an example of a value that is not matched:
    Binop (Mul, _, _)
  \end{lstlisting}

\end{frame}

\begin{frame}[fragile]
  \frametitle{An Extended Calculator}
  We extend our small language with a \textit{tenary} operator and
  replacing multiplication with  \textit{less equal} operator:

  \[(5 < 4) \mathbin? (4 + 4) : (5 + 5) \to 10\]

  \pause
  But now we can create illegal expressions:

  \[4 + (5 < 4) \to *error*\]

\end{frame}

\begin{frame}[fragile]
  \frametitle{An Extended Calculator}
  \lstinputlisting[basicstyle={\footnotesize\ttfamily}]{adtparser_tenary.ml}
\end{frame}

\begin{frame}[fragile]
  \frametitle{An Extended Calculator}
  The Compiler will warn us about problems:
  \begin{lstlisting}[basicstyle={\footnotesize\ttfamily},language=make]
    File "adtparser_tenary.ml", line 11, characters 10-24:
    Warning 8: this pattern-matching is not exhaustive.
    Here is an example of a value that is not matched:
    (Bool _, _)
    File "adtparser_tenary.ml", line 14, characters 10-24:
    Warning 8: this pattern-matching is not exhaustive.
    Here is an example of a value that is not matched:
    (Bool _, _)
    File "adtparser_tenary.ml", line 17, characters 10-16:
    Warning 8: this pattern-matching is not exhaustive.
    Here is an example of a value that is not matched:
    Int _
  \end{lstlisting}

\end{frame}

\begin{frame}[fragile]
  \frametitle{A Safe Calculator}
  Using Generalized Algebraic Data Types, we can express the
  restrictions in our algrbra.

  \lstinputlisting[basicstyle={\footnotesize\ttfamily}]{gadtparser.ml}
\end{frame}

\begin{frame}[fragile]
  \frametitle{A Safe calculator}
  The compiler will not accept illegal expressions:

  \[4 + (5 < 4)\]

  \begin{lstlisting}[language=,morekeywords={Lt,bool,int}]
    # Binop (Add, Int 4, Binop (Lt, Int 5, Int 4));;
    Error: This expression has type bool binop
    but an expression was expected of type int binop
    Type bool is not compatible with type int
  \end{lstlisting}

\end{frame}

\begin{frame}[fragile]
  \frametitle{AVL trees}
  Defining a balanced tree which the compiler will validate that its
  always balanced.

  \begin{lstlisting}
    (* Peano numbers: succ (succ (succ z)): 3 *)
    type z
    type succ
    type 'a s = succ * 'a

    type _ node =
      | LNode: ('a s node * 'a node   * int) -> 'a s s node
      | RNode: ('a node   * 'a s node * int) -> 'a s s node
      | BNode: ('a node   * 'a node   * int) -> 'a s node
      | Empty: z node
  \end{lstlisting}

\end{frame}

\section{At Scale}
\begin{frame}[fragile]
  \frametitle{OCaml at Scale}
  \begin{itemize}
  \item Using OCaml for more that 5 years at issuu
  \item Approx. 50.000 lines of OCaml code in production
  \item Microservice based architecture in an heterogenious environmentq
  \item Single thread parses 2.000/sec large json structure and send
    statistics to \textit{graphite}
  \item Created DSL for user behavioural analysis, analysing more than
    7 Billion events (one month) in (tenths of) minutes.
  \end{itemize}
\end{frame}

\begin{frame}[fragile]
  \frametitle{Caveats}
  \begin{itemize}
  \item No parallism
  \item d
  \end{itemize}
\end{frame}

\section{Try OCaml}
\begin{frame}[fragile]
  \frametitle{Using OCaml}
  \begin{itemize}
    \item Try online version of OCaml: \ttfamily{https://try.ocamlpro.com/}
    \item Read \emph{Real World OCaml} at \ttfamily{https://realworldocaml.org/}
    \item Install it locally using \emph{opam}, the OCaml package mananger.
  \end{itemize}
  \begin{lstlisting}[numbers=none,frame=none]
    $ brew install opam
    $ opam switch 4.02.3
    $ opam install utop
    $ utop
  \end{lstlisting}
\end{frame}

\begin{frame}[fragile]
  \begin{center}
    \LARGE Questions
  \end{center}
\end{frame}

\end{document}


%% Need to create a webserver, using the API to actually produce
%% something

%% Mention that ocaml cannot branch on type runtime. Therefore we
%% need to tag our variables - like units.

%% Show opam - That there exists a community
%% Second section is about not representing the illegal state.

%% Redo first section
%% - Use current example to show all places where the illegal state is
%% representable, and fix it.
